\section{Method}
\label{sec:method}
As \openpose{ } is only trained on humans standing, walking, running and sitting.
The assumption in this paper is that \openpose{ } is bad at humans on the ground.
But is that true or not?
If \openpose{ } is as good as a human at identifying where each feature on the human is located in both 2D and 3D, then a hypothesis formed such that the median and variance of both human and open pose should be equal.
$$
\text{H}_0\quad \mu_{human} = \mu_{OpenPose} \quad\text{AND}\quad \sigma_{human} = \sigma_{OpenPose}
$$
$$
\text{H}_1\quad \mu_{human} <> \mu_{OpenPose} \quad\text{OR}\quad \sigma_{human} <> \sigma_{OpenPose}
$$


% In this paper the proposed method to prove or disprove this hypothesis is to use a minimal data set of 21 images\ref{sec:appendices} each with several \aruco{ } corners.
% Those \aruco{ } corners are then used to determine the camera position relative to the corner at the head of the subject  in the scene.


% The test is done in both 2D and 3D with a minimal error method using \aruco corners.
% In the end, the 3D method did not work, but sufficient validation for the hypothesis was derivable from the 2D position.
% To prove the claim that \openpose{ } was unable to find the human features accurately.
% A F-test for testing the variance and a T-test for testing the median is performed on data that is derived both from several collected data points from both human and OpenPose.





