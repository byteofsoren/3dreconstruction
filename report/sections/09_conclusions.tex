\section{Conclusions}
\label{sec:conclusion}
In this report its attempted to validate \openposeS by comparing it to a human labeller.
The attempted solution is to reconstruct the positions for each label in 3D.
This is done by using a commination of technique's containing \arucoS corners, pose quivering and a Dijkstra method to solve the camera locations.
However the initial result of the output of the \openposS algorithm concludes that the variance is to crude falls the $H_0$ hypotheses, thus the accurate location of a feature could not be attained.
The camera location where also not successfully solved and in any case could probably not be solved completely correctly with out solving the bundle adjustment.

%I detta avsnitt ska du summera rapporten samt presentera slutsatser och slutanalys. Ge en kort översikt av syftet och frågeställningen. Du ska sedan tydligt tala om de viktigaste resultaten, förklara deras signifikans och sätta in dem i sitt sammanhang. Alla slutsatser ska ha stöd i tidigare delar av rapporten.  Du ska däremot inte presentera nya detaljer.

%En expert ska kunna läsa detta avsnitt oberoende av resten av rapporten.

