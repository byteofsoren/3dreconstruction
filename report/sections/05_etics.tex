\section{Ethics}
\label{sec:ethics}

%I först hand avser vi med ”etik” här forskningsetiska frågor. Innebär ditt val av frågeställning eller metod något forskningsetiskt ställningstagande? Om du till exempel intervjuar personer för ditt arbete, kan du garantera dessa anonymitet och på vilket sätt använder du den information du får av dem? Finns det andra etiska aspekter att beakta i arbetet? Kan det finnas etiska aspekter på resultatet av ditt arbete? Du bör tydligt ange om du anser att ditt arbete inte innehåller några forskningsetiska frågor.

%Du ska också kritiskt granska och analysera ditt arbete med hänsyn till samhälleliga aspekter. Här kan du till exempel diskutera hur ditt arbete förhåller sig till mål som ekonomisk, social och ekologiskt hållbar utveckling. Det kan också finnas juridiska och politiska aspekter på ditt arbete.
According to Starrett, Steve in \cite{starrett2017engineering} the engineer should consider un-ethical use of the implemented system's.
Thus in this part, the system is broken up into previously discussed parts of vision and the robot itself as each faeces different ethical considerations.
As this work uses images of actual humans, the risk is that a picture used in this work is used maliciously.

\subsection{Figures in this report}%
\label{sub:Figures_in_this_report}
Several figures in this report is generated by various OpenSource programs and compilers.
The \ac{uml} diagrams is according to\footnote{\url{https://plantuml.com/de/faq}} owned by the author of their corresponding sources code.
The tics figure is part of the LaTeX package and thus supported by GPL\@.
A few images are ether drawn in Incscape or created in freeCAD to get the 3D figures, thus also covered by GPL\@.
Finally the dataset is owned by the author and the results feature location graphics is created by the author.
Other than that, there are no primary ethical considerations for this work.