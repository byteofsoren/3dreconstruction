\section{Introduction}%
\label{sec:intro}
The motivation for this report the constant aging population in the western and eaten world.
This aging going up is a sign that the world are a bit more piece full, with out any major wars, hunger and general destruction.
According to OECD~\cite{oecd2018oecd}, the elderly population in Europe is about 20\% of population and Japan is on 27\%.
The increasing elderly population requires an increasing part of the younger population to take care of the elderly.
This could led to a downward spiral where as the numerous elderly increases more and more resources is needed to maintain a good living standard for the elderly, while less and less resources is created due to the younger generation is more and more occupied with taking care of the elders.

To counteract that future many companies is looking toward robotics and automations to find a solution.
Just searching for "Elderly" in a data base for technical research papers would give you over 7000 papers on the subject.
The authors Soren Tranberg et,al~\cite{Hansen2010evalrobot} also evaluated the use of robotics in healthcare already in 2010.
They conclude that the acceptance for robotics is high but considered that the products where unstable or complicated the concluding benefits was vague.

What drew the author to this subject is the challenging nature of creating such assist system.
Questions like how can that be done with out creating a system that are more expensive then a car.
Can sparse reconstruction really be done safely or is there any hidden limitations in the question it self.
But also soft questions like, would elderly trust the robot enough to dare to use it for its intended purpose?

The \ac{uma} is creating a robot that would have the task to to assist the elderly.
In this case the assist is to help an elderly subject on the flour to stand back up again with out calling for extra personal.
This is also not purl for economical reasons but also for a sens of self and a way to respect privacy of the subject.

% In their approach, they are set on using \openpose in a single camera configuration to find and assist the elderly.
\ac{uma}'s approach uses a robot arm with a single camera mounted on the gripper of the robot.
As this robot only have one camera mounted on the robot arm the 3D reconstruction of the subject needs to be completed by other methods then using binocular stereo vision as the robot in the paper~\cite{borangiu2010robot}.
One such solution is tho simply move the robot arm to two different position and the use same technique.
The draw back of that is the computational load for reconstructing the entire scene.
That is expensive and slow.

At \ac{uma} instead the approach is to use a preprocessing step to reduce the features in the image to a approximated skeleton of the subject.
Thus the expected outcome is to reduce several images in to its core components, the skeleton.
Then if each image contains its own skeleton the spars 3D reconstruction of the skeleton can be derived from those images.
However the above assumes that the camera position for each image is known and in this work that is not the case.
Instead this paper uses fiducial markers to identify the camera position for each image.

The question then is can a system that uses a single camera be used to do a sparse 3D reconstruction of the subject at hand and can that be done from all directions and angles?

The related works~\ref{sec:related_work} explore the \openpose algorithm for a way to do the sparse 2D reconstruction that later is used to do the space 3D reconstruction by using fiducial markers for the camera position is covered in Method~\ref{sec:method}.

% One such system that can be used to generate the skeleton of the subject is \opnpose algorithm.
% \openpose is a OpenSource product that uses part affinity fields with a \ac{cnn} to generate the skeleton from the subject in the image.
% To then reconstruction the sparse representation of the 3D skeleton using \openpose the intended system will use \aruco corners to determine the position of the camera in relation to the skeleton in the figure to generate the sparse 3D representation of the skeleton.

% This paper aims to answer if such system could be used to correctly classify the subjects limbs from all angles and what will happen if the spatial position derived by just using the input images.

% But in order to get an accurate description of the skeleton the input from the 2D skeleton needs to be sufficient.

% And the main problem with the approach of using \openpose is that it is trained on subjects that is standing, walking/running and sitting but the subjects in that \ac{uma} tries to assist are on the ground.
% This then is also covered by this report by using statistics on the results from the \openpose algorithm.










% Thus it is relevant to investigate how \openpose reacts if an elderly human is lying on the ground.
% robot is used to either assist or pick them up.
%In this thesis, an investigation to how OpenPose
% The problem is that OpenPose only seems to be trained on datasets where humans are standing, sitting, walking and running.

% Later, that was proven to be unsolvable due to a cumulative error when traversing the nodes.


% \subsection{Structure of this report}%
% \label{sub:Structure_of_this_report}
\textbf{Structure of the report}\\
The general structure of this document is as follows:
The Related Works in~\ref{sec:related_work} will discuss what others have done in this field of study.
Method~\ref{sec:method} derives the necessary steps and forms a hypothesis to solve the problem.
There is an ethics section~\ref{sec:ethics} that discusses the ethical part of this report and the licenses of the figures in the report.
Results~\ref{sec:results} show the results from the experiments and systems.
Discussion~\ref{sec:discussion} is the authors reflection on the results.
Conclusion~\ref{sec:conclusion} summarizes the findings of the report.
Appendices~\ref{sec:appendices} contains tables and images used in this report.


% The resulting data is a point cloud of each feature, formed by a median point algorithm explained in~\ref{sec:background}
% \label{sec:s3d:ntroduction}
% \section{Results}\label{sec:results}
% \label{sec:method}
% \label{sec:background}
%\label{sec:bg}
% \label{sec:s3d:Method}




% \par The expected outcome from this project presented in point in~\ref{sub:expected_outocem} show what kind of data will be acquired and how that is supposed to be stored.




% \subsection{Hypothesis and research questions}%
% \label{sub:Hypothesis}
% The hypothesis for this job is that \openpose can not correctly identify the features of humans lying on the floor from certain angles.
% Thus, the hypothesis for this part of the work is:
% \vspace{5mm}
% \begin{itemize}
%     \item[$\mathbf{H_0}$] The standard deviation of \openpose is on pair with a human from all angles.
%     \item[$\mathbf{H_a}$] The standard deviation of \openpose significantly deviates at certain angles.
% \end{itemize}

% \vspace{5mm}
% \raggedright Research questions used to prove the hypothesis above.
% \begin{itemize}
%     \item[$\mathbf{Q_{1}}$] How does the accuracy of \openpose change as the camera is moving to a position where the subject is upside down?
%     \item[$\mathbf{Q_2}$] How does \aruco handle multiple fiducial markers in one image?
%     % \item[$\mathbf{Q_3}$]
% \end{itemize}

% \def\svgwidth{\columnwidth}
% \begin{figure}[ht]
%     \centering
%         % \input{Figures/validation.pdf_tex}
%         % \input{Figures/validation.pdf_tex}
%         \input{images/dataset.pdf_tex}
%       \caption{
%                       In this figure, the subject is on its back facing upward and from the for angles in the figure the two features marked as $F_{sr}$, $F_{er}$ for right shoulder and elbow respectively.
%           In the last image, its proposed the elbow can not be found from that camera angle thus not marked.
%           The \arucos  corner at the head is the origin for the system with corner 0.
%           Also for the last image, the camera cannot observe the origin, but as each other \arucos  corners is observed from the other angle, the camera position can still be calculated from the other \arucos  corners.
%       }
%       \label{fig:intro:dataset}
% \end{figure}


% \subsection{Expected outcome}%
% \label{sub:expected_outocem}
% The expected outcome from this work is a dataset that contains the following data.
% \begin{itemize}
%     \item Several sets each containing images of one subject on the ground.
%     \item In each set, there are several images from several angles.
%     \item 2D annotated data from the photos using both a human and \openpose
%     \item 3D pose cloud, estimated from each feature from human data and \openpose
% \end{itemize}

% % In this proposal, the data is proposed to be stored as CSV and YAML  files.
% % The CSV data contains the 2D pose for each set of images, while the YAML file contains settings for that set of data.

% The figure~\ref{fig:intro:dataset} contains a supposed set of images from the dataset with two notable features in the image.












