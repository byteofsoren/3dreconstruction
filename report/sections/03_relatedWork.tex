\section{Related Work}
\label{sec:related_work}
In this paper, an evaluation of accuracy is performed on subjects lying on the ground from several angles with body parts partly occlusion because it remains hidden under the body.
This occlusion can pose a problem for methods that uses a model-based/top-down approach mentioned by Sarafianos et al.~\cite{sarafianos2016} with a pre-defined skeleton as suddenly the images have a missing body part due to self-occlusion.
Future more the according to Sarafianos, the top-down approach takes significantly more time to compute than a generative bottom-up model.
%But the bottom line is that pose estimation from a monocular camera requires heavy computation for each frame.
Sins machine learning is the primary method for solving this problem; a god data set is necessary.

% walking
In the broader perspective, Erika D’Antonio et al. \cite{d2021validation} evaluates the accuracy of \openposeS when the subject is walking or running by observing the joint angles between each joint.
That was done by using two fixed cameras, a \ac{imu} and linear triangulation algorithms. It was performed on six healthy subjects.
Among the results, a significant effect of the camera positioning concerning the subject was observed.
Thus, the analysed results focused on the accuracy of the hip and knee joints from one camera direction.
To differentiate the results, the authors use statistical methods to conclude that the \operpose{ } is not adequate for a complete human body kinematics analysis.

% On the ground
Currently, at the time of writing, there exist no 3D pose dataset for humans that are lying on the ground outside the lab environments \cite{yang2018, mehta2017, yasin2016, wang2019}.
However, in late 2020, a new paper with a set of images with humans laying in beads was released by Liu et.al\cite{liu2020simultaneously}, thus laying the groundwork for starting a set where the subject is laying down.
This set was named \ac{slp} with includes 109 participant S lying in a bed.
The set then contains images taken with an \ac{rgb} camera, \ac{rgbd} camera with depth information, \ac{lwir} camera and a pressure mat.
This paper aims to create a dataset with subjects lying on a bed with our without a blanket.
Thus the results from that paper are not directly aimed at the validation of the system.

The authors Sun, Guangmin and Wang, Zhongqi~\ref{sun2020falldetect} attempts to detect if an elderly human falls on to the floor.
This is done by using the \opnepose algorithms to detect if the subject is on the ground.
The authors tries to deal with the false positives from \openpose using a object classifier named SSD-MobilNet.
\todo[inline]{Write more}


% Slam navigation.
%To accurately reconstruct 3D geometry, a camera pose estimation method is used.
Camera pose estimation is commonly used in robotics and \ac{vr} to determine the 3D dimensional position of either the player or the robot.
To to solve this problem according to Mu{\~n}oz-Salinas et.al\cite{munoz2018mapping} a great part of the research focus on \ac{sfm} and \ac{slam}.
However, Mu{\~n}oz-Salinas continues, keypoint matching has a somewhat limited invariability to scale and rotation, which can make the method unable to find a solution.
In their work, they divide the tasks into three steps, mapping the markers in the input domain, creating a pose quiver for how each marker is connected, and finally calculating each marker's global position and camera using the shortest path algorithm.
Nevertheless, they refine the position by solving the bundle adjustment problem to fine calibrate the position.
These results it is shown that they achieve better maps and localisation than native \ac{slam} or \ac{sfm} under a broader range of viewports.

%aruco markers and aruco mapping
% reliable markers \cite{garrido2014automatic}
% Mapping and localization from planar markers \cite{munoz2018mapping}
% Camera pose estimation using aruco
% Camera pose estimation is a common problem in many \ac{ar} applications.
% In~\cite{garrido2014automatic} the authors Garrido-Jurado et al discusses there contributions with the freely available\footnote{Under BSD license.} \aruco{ } library they created.


