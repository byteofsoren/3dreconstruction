% ============================= Abstract ==============================
\begin{abstract}
Det här avsnittet ska helt enkelt vara just detta: en sammanfattning av hela rapporten. En lämplig omfattning är c:a 200 – 250 ord.  En bra tumregel är att sammanfattningen ska hållas så kort det går, den ska vara kompakt men fortfarande tydlig, informativ och väcka intresse. Ge de viktigaste fakta
och summera allt det som är väsentligt i rapporten.  Följande bör ingå:
\begin{itemize}
\item[--]	Presentation/introduktion av området för arbetet
\item[--]	översiktlig presentation av uppgiften inklusive syfte och frågeställning
\item[--]	Motivation till varför området och uppgiften är viktiga och intressanta
\item[--]	Generell beskrivning av hur du angripit uppgiften, vad du har gjort
\item[--]	Sammanfattning av resultat och slutsatser och vad ditt arbete bidrar med
\end{itemize}

Inga detaljer ska vara med i sammanfattningen, inte heller beskrivning av hur rapporten är uppställd. 
Sammanfattningen ska kunna läsas helt fristående från resten av rapporten, och av en ganska bred grupp av läsare. Den ska ge en bra grund för att en läsare ska kunna bedöma om hen är intresserad av att läsa hela rapporten. 
Sammanfattningen är den del av en rapport som läses allra mest och av flest personer. Därför är det extra viktigt att du skriver en bra sammanfattning. Du behöver ha ett ordentligt grepp om innehållet i rapporten när du skriver sammanfattningen, och när hela rapporten är klar bör du granska och vid behov revidera sammanfattningen så att den överensstämmer med rapporten.

\end{abstract}